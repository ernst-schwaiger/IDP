\section{Threat identification}
\label{chapter5}

\subsection{Identified threats and countermeasures}


	\begin{enumerate}
		\item \textbf{Threat 1:} Manipulation
            
            \textbf{Description:} An attacker could attempt to alter the messages transmitted between Node 1 and Node 2. This could result in false sensor data or control commands being introduced, leading to dangerous situations (e.g., incorrect distance = incorrect speed control).
            
            \textbf{Solution:} Data integrity is ensured by an HMAC (hash-based message authentication code).
%			Requirement see \ref{req.1.1} \\
        
        \paragraph{} 
		\item \textbf{Threat 2:} Denial of Service (DoS)
            
            \textbf{Description:} An attacker or jammer could block the Bluetooth connection (e.g., through jamming or flooding). This would render sensor data unavailable and cause the ACC to lose its basis.
            
            \textbf{Solution:} Implementation of a timeout mechanism for Bluetooth messages. If the connection fails for longer than a defined period of time, the system automatically switches to manual operation.
%			Requirement see \ref{req.1.1} \\
        
        \paragraph{} 
		\item \textbf{Threat 3:} Replay-Attack
            
            \textbf{Description:} An attacker could resend previously intercepted messages to Node 2 (replay). This could allow outdated but formally valid distance data to be used, leading to dangerous rule decisions.
            
            \textbf{Solution:} Use of a sequence counter or sequential message number in each packet.
%			Requirement see \ref{req.1.1} \\
        
        \paragraph{} 
	\end{enumerate}
	
\subsection{Identified threats without countermeasures}

\begin{enumerate}
		\item \textbf{Threat 1:} Preshared Key not protected

            \textbf{Description:} The preshared key is stored openly in the repository. If accessed by unauthorized persons, it could be used to impersonate trusted nodes. This is acceptable for the prototype stage, but proper key storage should be implemented in production.
            
        \paragraph{}
		\item \textbf{Threat 2:} No protection against buffer overflows
        
            \textbf{Description:} The code does not include checks to prevent buffer overflows. This could allow memory corruption or program crashes. As the focus lies on functionality, no memory-protection mechanisms are applied.

        \paragraph{}
		\item \textbf{Threat 3:} Rowhammer attacks not considered
        
            \textbf{Description:} Hardware-level Rowhammer attacks are not mitigated in this setup.


	\end{enumerate}
